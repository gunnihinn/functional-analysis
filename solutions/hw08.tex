\documentclass[11pt]{article}

\usepackage{tgpagella}
\linespread{1.0}
\usepackage[utf8]{inputenc}
\usepackage[T1]{fontenc}

\usepackage[normalem]{ulem}
\usepackage{textcomp}
\usepackage{hyperref}

\usepackage{amsmath}
\usepackage{amssymb}
\usepackage{amsthm}

\usepackage{tikz-cd}
\usepackage{color}

\newtheorem*{theo}{Theorem}
\theoremstyle{definition}
\newtheorem{prob}{Problem}
\newtheorem*{defi}{Definition}

\newcommand{\kk}[1]{\mathbf{#1}}
\newcommand{\cc}[1]{\mathcal{#1}}

\def\<{\langle}
\def\>{\rangle}
\def\norm#1{\| #1 \|}

\DeclareMathOperator{\Hom}{Hom}
\DeclareMathOperator{\supp}{supp}
\DeclareMathOperator{\im}{im}
\DeclareMathOperator{\codim}{codim}
\DeclareMathOperator{\ann}{ann}

\def\head{
	\begin{center}
		\textbf{\LARGE Homework set 8}
		\\[3pt]
		Due by 15:00 on Monday, October 16, 2023.
	\end{center}
	\medskip
}

\begin{document}

\head

Please select \textbf{three} problems to solve and hand in written solutions
either in person or to \verb+gunnar@magnusson.io+.

\begin{prob}
	Let $V$ be a Hilbert space.
	Show that $V^\vee$ is a Hilbert space when given the inner product
	\[
		\< f, \bar g \> = \< y, \bar x \>,
	\]
	where $x, y \in V$ and $x \mapsto f$ and $y \mapsto g$ under the Riesz representation theorem.
\end{prob}

\begin{proof}[Solution]
	Let's denote this correspondence by $x = \rho(f)$.
	Then we have $f(z) = \< z, \bar x \> = \<z, \overline{\rho(f)}\>$ for all $z \in V$.
	We claim that $\<f, \bar g\> = \<\rho(g), \overline{\rho(f)}\>$ defines an inner product:
	It is additive in each variable because $\rho$ is additive, and we have $\rho(\lambda f) = \bar \lambda \rho(f)$, so it is sesquilinear.
	Further $|f|^2 = |\rho(f)|^2 = 0$ if and only if $\rho(f) = 0$, which happens if and only if $f = 0$ by Riesz.

	Suppose then that $(f_n)$ is a Cauchy sequence in $V^\vee$,
	and set $x_n = \rho(f_n)$.
	As $|f_n - f_m| = |x_n - x_m|$ by definition, it follows that $(x_n)$ is Cauchy, so its limit $x$ exists.
	Set $f(y) := \<y, \bar x\>$.
	Then $|f - f_n| = |x - x_n|$ so $f_n \to f$.
\end{proof}

\begin{prob}
	Recall that the annihilator of a subset $M \subset V$ of a normed space is the set
	\[
		\ann M = \{ f \in V^\vee \mid \text{$f(x) = 0$ for all $x \in M$}\}.
	\]
	Discuss the relationship between $M^\perp$ and $\ann M$ in a Hilbert space $V$.
\end{prob}

\begin{proof}[Solution]
	There is a linear map
	\[
		\theta : M^\perp \to \ann M,
		\quad
		y \mapsto (x \mapsto \<x, \bar y\>).
	\]
	It is injective by construction, and bounded as $|\theta(x)| = |x|$ for all $x$.
	If $f \in \ann M$ there is a $y \in V$ such that $f(x) = \<x, \bar y\>$ for all $x$ by Riesz.
	But for $x \in M$ we have $0 = f(x) = \<x, \bar y\>$, so $y \in M^\perp$, and $\theta$ is thus surjective.
\end{proof}

\begin{prob}
	Let $V$ be a Hilbert space and let $S \subset V$ be a closed subspace.
	Show that the quotient space $V/S$ is also a Hilbert space.
	Show that $S^\perp$ is isometric to $V / S$.
\end{prob}

\begin{proof}[Solution]
	By Homework 6, Problem 4 we know that $(V/S)^\vee \cong \ann S$.
	Problems 1 and 2 here then give us a diagram
	\[
		\begin{tikzcd}
			V/S \ar[d] & S^\perp \ar[d]
			\\
			(V/S)^\vee \ar[r] & \ann S
		\end{tikzcd}
	\]
	where all the arrows are isometries, so we win an isometry $V/S \to S^\perp$.
\end{proof}

\begin{prob}
	Let $V$ and $W$ be Hilbert spaces and $(f_n)$ a sequence of bounded operators from $V$ to $W$ such that $f_n \to f$.
	Show that $f_n^* \to f^*$.
\end{prob}

\begin{proof}[Solution]
	First let $f : V \to W$ be an arbitrary bounded operator.
	For any $x \in V$ and $y \in W$ we have
	\[
		\< f(x), \bar y \> = \< x, \overline{f^*(y)} \>
	\]
	and we then see that
	\[
		|f^*(y)|^2
		= \< f^*(y), \overline{f^*(y)} \>
		= \< f(f^*(y)), \bar y \>.
	\]
	Now
	\[
		|\< f(f^*(y)), \bar y \>|^2
		\leq |f(f^*(y))|^2 |y|^2
		\leq |f|^2 |f^*(y)|^2 |y|^2
	\]
	so taking square roots and remembering that the inner product term was nonnegative we see that
	\[
		|f^*(y)|
		\leq |f| |y|
	\]
	for all $y$
	and thus $|f^*| \leq |f|$.

	Now, by linearity we may assume that $f = 0$, and then $f^* = 0$.
	By the above we then get $|f_n^*| \leq |f_n| \to 0$.
\end{proof}

\begin{prob}
	Let $V$ and $W$ be Hilbert spaces and $f : V \to W$ a bounded operator.
	Show that:
	\begin{enumerate}
		\item $\im f^* \subset (\ker f)^\perp$.
		\item $(\im f)^\perp \subset \ker f^*$.
		\item $\ker f = (\im f^*)^\perp$.
	\end{enumerate}
\end{prob}

\begin{proof}[Solution]
	1. Let $y \in W$ and $x \in \ker f$. Then
	\[
		0 = \< f(x), \bar y \> = \< x, \overline{f^*(y)} \>
	\]
	so $f^*(y) \in (\ker f)^\perp$.

	2. Let $x \in V$ and $y \in (\im f)^\perp$. Then
	\[
		0 = \< f(x), \bar y \> = \< x, \overline{f^*(y)} \>.
	\]
	But $x$ was arbitrary so we can take $x = f^*(y)$ and see that $f^*(y) = 0$, so $y \in \ker f^*$.

	3. From 1 we see that $\ker f \subset (\im f^*)^\perp$.
	Applying 2 to $f^*$ we get that $(\im f^*)^\perp \subset \ker (f^*)^* = \ker f$.
	Together we get the result.
\end{proof}

\begin{prob}
	Let $V$ and $W$ be Hilbert spaces and $f : V \to W$ a bounded operator.
	Show that $f^* f : V \to V$ is a bounded self-adjoint operator and that $|f| = \sqrt{|f^* f|}$.
\end{prob}

\begin{proof}[Solution]
	Since $f$ is bounded then so is $f^*$, and the composition of bounded operators is bounded. Now
	\[
		|f|
		= \sup_{x\not=0} \frac{|f(x)|}{|x|}
		= \sup_{x\not=0} \frac{\sqrt{\<f(x), \overline{f(x)}\>}}{|x|}
		= \sup_{x\not=0} \frac{\sqrt{\<f^*f(x), \bar x}\>}{|x|}
		\leq \sqrt{|f^*f|}
	\]
	by Cauchy--Schwarz.
	On the other hand we have $|f^*f| \leq |f^*||f| \leq |f|^2$, so $|f^*f| = |f|^2$.
\end{proof}


\end{document}







































































