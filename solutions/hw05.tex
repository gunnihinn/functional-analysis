\documentclass[11pt]{article}

\usepackage{tgpagella}
\linespread{1.0}
\usepackage[utf8]{inputenc}
\usepackage[T1]{fontenc}

\usepackage[normalem]{ulem}
\usepackage{textcomp}
\usepackage{hyperref}

\usepackage{amsmath}
\usepackage{amssymb}
\usepackage{amsthm}

\usepackage{tikz-cd}
\usepackage{color}

\newtheorem*{theo}{Theorem}
\theoremstyle{definition}
\newtheorem{prob}{Problem}
\newtheorem*{defi}{Definition}

\newcommand{\kk}[1]{\mathbf{#1}}
\newcommand{\cc}[1]{\mathcal{#1}}

\def\<{\langle}
\def\>{\rangle}

\DeclareMathOperator{\Hom}{Hom}
\DeclareMathOperator{\supp}{supp}
\DeclareMathOperator{\im}{im}

\def\head{
\begin{center}
\textbf{\LARGE Homework set 5}
\end{center}
\medskip
}

\begin{document}

\head

\begin{prob}
If $V$ is a normed space and $S \subset V$ a subset, we define the distance from
a point $x$ to $S$ by $d(x, S) = \inf_{y \in S} |x - y|$.
Let $f : V \to \kk R$ be a bounded linear functional and let
\[
H_f = \{ x \in V \mid f(x) = 1 \}.
\]
Show that
\[
|f| = \frac{1}{d(0, H_f)}.
\]
\end{prob}

\begin{proof}[Solution]
If $|f| = 0$ then $f = 0$ and $H_f = \varnothing$ and $\inf$ of the empty set
is $\infty$, so that works.
We may thus assume that $f \not= 0$.

If $f(x) = 0$ then $|f(x)|/|x| = 0$, so
\begin{align*}
|f|
= \sup_{x\not=0} \frac{|f(x)|}{|x|}
= \sup_{f(x)\not=0} \frac{|f(x)|}{|x|}
&= \sup_{f(x)\not=0} \frac{1}{|x/f(x)|}
\\
&= \frac{1}{\inf\limits_{f(x)\not=0} |x/f(x)|}
= \frac{1}{d(0, H_f)}.
\qedhere
\end{align*}
\end{proof}

\begin{prob}
Let $V = \kk R^2$ and let $f : V \to V$ be the operator defined by the matrix
\[
A = \begin{pmatrix}
1 & 2 \\ 3 & 4
\end{pmatrix}.
\]
Calculate $|A|$ when $V$ has each of the norms below:
\begin{enumerate}
\item
$|(x,y)|_1 = |x| + |y|$.

\item
$|(x,y)|_2 = \sqrt{x^2 + y^2}$.

\item
$|(x,y)|_\infty = \max\{|x|, |y|\}$.
\end{enumerate}
If this is too difficult we may also accept bounds in terms of the other norms
or numerical approximations.
\end{prob}

\begin{proof}[Solution]
Case by case analysis shows the answer in 1 is 6, and the answer in 3 is 7.
The answer in 2 is the largest square root of an eigeinvalue of $A^t A$,
or approximately 5.465 if we don't want to calculate that.
\end{proof}

\begin{prob}
Let $V$ and $W \not= 0$ be normed spaces.
Suppose that $\dim V = \infty$.
Show that there exists an unbounded linear operator $f : V \to W$.
Conclude that $V^\vee \not= V^*$.
(Hint: Every vector space has a basis.)
\end{prob}

\begin{proof}[Solution]
Let $(e_\alpha)_{\alpha \in J}$ be a basis for $V$.
As $\dim V = \infty$ the set $J$ contains a countable subset $J_0$ that
we identify with $\kk N$.
Pick a nonzero $y \in W$ of norm $1$ and define $f : V \to W$ by
\(
f(e_\alpha) = n y
\)
if $n \mapsto \alpha$ and $0$ otherwise, and extend by linearity to all of $V$.
Then $f$ is an unbounded linear operator.
We can in particular do this when $W = \kk R$, so there always exists an
unbounded linear functional on an infinite-dimensional space.
\end{proof}


\begin{theo}[Hahn--Banach]
Let $V$ be a normed space and let $S \subset V$ be a subspace.
If $f \in S^\vee$ there exists a bounded extension $\hat f \in V^\vee$ of $f$
such that $|\hat f| = |f|$.
\end{theo}

\begin{prob}[Proof of Hahn--Banach, part 1]
Let $V$ be a normed space.
\begin{enumerate}
\item
Let $S \subset V$ be a linear subspace.
Let $x_1 \in V \setminus S$ and let $S_1 = S + \kk R x_1$.
Show that $S_1$ is a linear subspace that contains $S$, and that every vector
in $S_1$ can be written uniquely as $x = x_0 + \lambda x_1$, where $x_0 \in S$
and $\lambda \in \kk R$.

\item
Let $f \in S^\vee$.
Show that $f_1(x) = f(x_0)$ is a bounded extension of $f$ from $S^\vee$ to
$S_1^\vee$, and that $|f_1| = |f|$.
\end{enumerate}
\end{prob}

\begin{proof}[Solution]
1. Obvious because $S_1 = S \oplus \kk R x_1$.

2. Define $f_1(x + \lambda x_0) = f(x) + \lambda c$, where $c$ will be chosen
later. This functional is clearly an extension of $f$ to $S_1$.
We would like it to satisfy $|f_1| = |f|$, and may assume that $|f| = 1$.
We would then like to have
\[
- |x + \lambda x_0| \leq f(x) + c \lambda \leq |x + \lambda x_0|
\]
for any $x$ and $\lambda$.
This holds for $\lambda = 0$ by hypothesis.
For $\lambda \not= 0$ we can rewrite this as
\[
- |x/\lambda + x_0| - f(x / \lambda)
\leq c
\leq |x/\lambda + x_0| - f(x / \lambda),
\]
or equivalently
\[
- |y + x_0| - f(y)
\leq c
\leq |y + x_0| - f(y)
\]
for $y \in S$.
For $y, z \in S$ we have
\[
f(z) - f(y)
= f(z-y)
\leq |z - y|
\leq |z + x_0| + |y + x_0|
\]
by $|f| = 1$, so in fact
\[
- |y + x_0| - f(y)
\leq |z + x_0| - f(z)
\]
for all $y, z \in S$.
Then
\[
a := \sup_{y \in S} -|y + x_0| - f(y),
\quad
b := \inf_{z \in S} |z + x_0| - f(z)
\]
are finite and satisfy $a \leq b$, so any $c \in [a, b]$ will do.
\end{proof}

\emph{A partial order} on a set $S$ is a binary relation $\prec$
on $S$ that satisfies:
\begin{itemize}
\item
Reflexivity: $x \prec x$.

\item
Antisymmetry:
$x \prec y$ and $y \prec x$ imply $x = y$.

\item
Transitivity:
$x \prec y$ and $y \prec z$ imply $x \prec z$.
\end{itemize}
As an example, consider the inclusion $U \subset V$ of subsets of $S$.

A subset $T \subset S$ is totally ordered if for every $x,y$ in $T$ we have
either $x \prec y$ or $y \prec x$.
An element $y$ is an upper bound for $T$ if $x \prec y$ for every $x \in T$.
Finally an element $y \in S$ is maximal if $x \prec y$ implies $x = y$.

Zorn's lemma says that if $(S,\prec)$ is a partially ordered set such that
every totally ordered subset contains an upper bound, then $(S,\prec)$ contains
at least one maximal element.


\begin{prob}[Proof of Hahn--Banach, part 2]
Let $V$ be a normed space and $S \subset V$ a subspace.
Let also $f \in S^\vee$ be a bounded linear functional.
Denote by $\cc L$ the set of all bounded extensions $(M, g)$ of $f$ to a
subspace $M$ such that $|g| = |f|$.

\begin{enumerate}
\item
Show that $(S, f) \in \cc L$, so it is not empty.

\item
We write $(M,g) \prec (M',g')$ if $M \subset M'$ and $g'(x) = g(x)$ for all $x
\in M$.
Show that this is a partial order on $\cc L$.

\item
Suppose that $\cc F$ is a totally ordered subset of $\cc L$.
Define a set $W = \bigcup_{(M,f) \in \cc F} M$.
Show that $W$ is in fact a vector subspace of $V$.

\item
Suppose that $\cc F$ is a totally ordered subset of $\cc L$.
Define $W$ as above and define $h : W \to \kk R$ by
\(
h(x) = g(x)
\)
for any $(M,g)$ such that $x \in M$.
Show that this is well-defined, and that $h$ is an extension of $f$.

\item
Conclude that there exists a maximal extension $h : W \to \kk R$ of $f$,
and use part 1 to conclude that we must have $W = V$.
\end{enumerate}
\end{prob}

\begin{proof}[Solution]
\end{proof}

\end{document}
