\documentclass[11pt]{article}

\usepackage{tgpagella}
\linespread{1.0}
\usepackage[utf8]{inputenc}
\usepackage[T1]{fontenc}

\usepackage[normalem]{ulem}
\usepackage{textcomp}
\usepackage{hyperref}

\usepackage{amsmath}
\usepackage{amssymb}
\usepackage{amsthm}

\usepackage{tikz-cd}
\usepackage{color}

\newtheorem*{theo}{Theorem}
\theoremstyle{definition}
\newtheorem{prob}{Problem}
\newtheorem*{defi}{Definition}

\newcommand{\kk}[1]{\mathbf{#1}}
\newcommand{\cc}[1]{\mathcal{#1}}

\def\<{\langle}
\def\>{\rangle}
\def\norm#1{\| #1 \|}

\DeclareMathOperator{\Hom}{Hom}
\DeclareMathOperator{\supp}{supp}
\DeclareMathOperator{\im}{im}
\DeclareMathOperator{\id}{id}
\DeclareMathOperator{\codim}{codim}
\DeclareMathOperator{\ann}{ann}

\def\head{
\begin{center}
\textbf{\LARGE Homework set 7}
\end{center}
\medskip
}

\begin{document}

\head

Recall that if $X$ is an inner product space and $Y \subset X$ is a complete
subspace then for any $x \in X$ there exists a unique $y \in Y$ such that $|x -
y| = d(x, Y)$.

\begin{prob}
Let $X$ be a Hilbert space and let $Y \subset X$ be a closed subspace.
Define $P : X \to X$ by $x \mapsto y$, where $y \in Y$ is the unique element
such that $|x - y| = d(x, Y)$.
\begin{enumerate}
\item
Show that $P$ is linear.

\item
Show that $P$ is bounded.

\item
Show that $P^2 = P$.

\item
Show that $\ker P = Y^\perp$ and $\im P = Y$.

\item
Conclude that $X = Y \oplus Y^\perp$.
\end{enumerate}
\end{prob}

\begin{proof}[Solution]
1.
We know that if $y = P(x)$ then $x - y$ is orthogonal to $Y$.
The converse is also true: If $y \in Y$ is such that $x - y \perp Y$ then $y =
P(x)$.
This holds because for such a $y$ and any $z \in Y$ we have
\[
|x - z|
= |(x - y) + (y - z)|
= |x - y| + |y - z|
\geq |x - y|.
\]
Suppose now that $x_1, x_2 \in X$ and let $y_j = P(x_j)$.
For $\lambda, \mu \in \kk C$ we then have
\[
\< \lambda x_1 + \mu x_2 - (\lambda y_1 + \mu y_2), \bar y \>
= \lambda \< x_1 - y_1, \bar y \>
+ \mu \< x_2 - y_2, \bar y\>
= 0
\]
for any $y \in Y$.
Therefore $\lambda P(x_1) + \mu P(x_2) = P(\lambda x_1 + \mu x_2)$.

2. Note that $d(x, Y) \leq |x|$ for any $x \in X$.
Then
\[
|P(x)|
= |P(x) - x + x|
\leq |P(x) - x| + |x|
\leq 2|x|.
\]

3. It is enough to show that $P(y) = y$ for any $y \in Y$.
But $d(y, Y) = 0$ for $y \in Y$, so $P(y) = y$.

4. If $y \in Y$ then $P(y) = y$, so $\im P = y$.
Let $x \in Y^\perp$.
Then $0 \in Y$ is such that $x - 0 = x \perp Y$, so $P(x) = 0$.
Now let $x \in \ker P$.
Then $x - P(x) = x$ is orthogonal to $Y$, so $x \in Y^\perp$.

5. For any $P : X \to X$ with $P^2 = P$ we have $X = \ker P \oplus \im P = Y
\oplus Y^\perp$.
\end{proof}

\begin{prob}
Let $X$ be a Hilbert space and $M \subset X$ a subset.
\begin{enumerate}
\item
Show that $M^\perp$ is a closed subspace of $X$.

\item
Show that $M \subset (M^\perp)^\perp$.

\item
Show that $M = (M^\perp)^\perp$ if $M$ is a closed subspace.
\end{enumerate}
\end{prob}

\begin{proof}[Solution]
1. $M^\perp$ is a subspace by linearity of the inner product.
It is closed because $M^\perp = \bigcap_{y \in M} \ker(x \mapsto \<x, \bar y\>)$.

2. Let $x \in M$ and $y \in M^\perp$.
Then $\<y, \bar x\> = \overline{\<x, \bar y\>} = 0$ so $x \in (M^\perp)^\perp$.

3. By Problem 1 we have $X = M \oplus M^\perp = M^\perp \oplus (M^\perp)^\perp$
and $M \subset (M^\perp)^\perp$.
We get exact sequences
\[
\begin{tikzcd}
0 \ar[r] & M \ar[d] \ar[r] & X \ar[r] \ar[d] & M^\perp \ar[r] \ar[d] & 0
\\
0 \ar[r] & (M^\perp)^\perp \ar[r] & X \ar[r] & M^\perp \ar[r] & 0
\end{tikzcd}
\]
where the second and third downward arrows are isomorphisms, so the first one
is as well by general theory.
\end{proof}

\begin{prob}
Let $X$ be a Hilbert space and $Y \subset X$ a subspace (not necessarily closed).
Show that $Y$ is dense in $X$ if and only if $Y^\perp = \{ 0 \}$.
\end{prob}

\begin{proof}[Solution]
Suppose $Y$ is dense.
If $x \in Y^\perp$ then there are $y_n$ in $Y$ such that $y_n \to x$.
But then $|x|^2 = 0$ by continuity.

Now suppose that $Y^\perp = \{0\}$.
It's clear that if $Y \subset Z$ then $Z^\perp \subset Y^\perp$, so
${\overline Y}^\perp \subset Y^\perp = \{ 0 \}$.
As $\overline Y$ is a closed subspace we then get
\[
\overline Y
= ({\overline Y}^\perp)^\perp
= \{ 0 \}^\perp
= X.
\qedhere
\]
\end{proof}

\begin{prob}
Let $X$ be an inner product space and fix $y \in X$.
Show that $f(x) = \<x, \bar y \>$ is a bounded linear functional on $X$
and that $|f| = |y|$.
\end{prob}

\begin{proof}[Solution]
This is a linear functional because the inner product is linear in its
first variable, and Cauchy--Schwarz gives
\[
|f(x)| = |\<x, \bar y\>| \leq |x| |y|,
\]
so $f$ is bounded and $|f| \leq |y|$.
By taking a multiple of $y$ we get equality, so $|f| = |y|$.
\end{proof}

\begin{prob}
Let $X$ be a Hilbert space and let $f \in X^\vee$ be a bounded linear
functional.
\begin{enumerate}
\item
Show that $X = \ker f \oplus \ker f^\perp$.

\item
Show that $\dim \ker f^\perp = 1$.

\item
Prove the Riesz representation theorem:
There exists a unique $y \in X$ such that $f(x) = \< x, \bar y \>$ for all $x$.
\end{enumerate}
\end{prob}

\begin{proof}[Solution]
1. As $f$ is bounded then $\ker f$ is a closed subspace, so by Problem 1 we have
$X = \ker f \oplus \ker f^\perp$.

2. We do have to assume that $f \not= 0$.
Let $y,z \in \ker f^\perp$ and suppose that both are nonzero.
By scaling we may assume that $f(y) = f(z) = 1$.
Then $y - z \in \ker f$, so
$0 = \<y, \overline{y - z}\> = |y|^2 - \<y, \bar z\>$
and $\<y, \bar z\> = |y|^2$.
Similarly we find that $\<z, \bar y\> = |z|^2$.
Multiplying together we get
\[
|\<y, \bar z\>|^2 = |y|^2 |z|^2,
\]
which only holds when $z = \lambda y$ by Cauchy--Schwarz.
Therefore $(y)$ is a basis of $\ker f^\perp$.

3.
We first show that there exists a nonzero $y \in \ker f^\perp$ such that
$f(y) = |y|^2$.
Pick any nonzero $y \in \ker f^\perp$.
Multiplying it by $e^{i\theta}$ we may assume that $f(y) > 0$.
Then multiplying by a real $t > 0$ we want to solve $t f(y) = t^2 |y|^2$
for $t$, which we can.

Now every $x \in X$ can be written as $x = z + \lambda y$, where $z \in \ker f$.
Then
\[
f(x) = \lambda f(y) = \lambda |y|^2 = \<x, \bar y\>.
\]
If $z$ is another such element, then we get $\<x, \bar y\> = f(x) = \<x, \bar
z\>$ for all $x$, so $\<x, \overline{y - z}\> = 0$ for all $x$.
Then $|y - z|^2 = 0$, so $z = y$.
\end{proof}

\end{document}
