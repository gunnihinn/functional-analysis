\documentclass[11pt]{article}

\usepackage{tgpagella}
\linespread{1.0}
\usepackage[utf8]{inputenc}
\usepackage[T1]{fontenc}

\usepackage[normalem]{ulem}
\usepackage{textcomp}
\usepackage{hyperref}

\usepackage{amsmath}
\usepackage{amssymb}
\usepackage{amsthm}

\usepackage{tikz-cd}
\usepackage{color}

\theoremstyle{definition}
\newtheorem{prob}{Problem}

\newcommand{\kk}[1]{\mathbf{#1}}
\newcommand{\cc}[1]{\mathcal{#1}}

\def\<{\langle}
\def\>{\rangle}

\DeclareMathOperator{\supp}{supp}

\def\head{
\begin{center}
\textbf{\LARGE Homework set 3}
\end{center}
\medskip
}

\begin{document}

\head


\begin{prob}
Let $V$ be a normed vector space.
Show that $E \subset V$ is bounded in the metrizable sense
if and only if it is bounded in the usual sense, that is, there exists
$r > 0$ such that $|x| \leq r$ for all $x \in E$.
\end{prob}

\begin{proof}[Solution]
Suppose $E$ is bounded in the metrizable sense.
Take $U = B(1)$.
There is a $\lambda > 0$ such that $E \subset \mu B(1)$ for any $\mu \geq \lambda$.
But then $|x| \leq |\lambda|$ for any $x \in E$.

Suppose then that $E$ is bounded in the norm sense.
Then there exists $r > 0$ such that $E \subset B(r)$.
Let $U$ be an open neighborhood around $0$.
Then there exists $\varepsilon > 0$ such that $0 \in B(\varepsilon) \subset U$.
It follows that $E \subset (\mu/\varepsilon) B(\varepsilon)
\subset (\mu/\varepsilon) U$ for any $\mu \geq r$.
\end{proof}

\begin{prob}
Let $V$ be finite-dimensional and let
$| \, \cdot \, |_1$ and
$| \, \cdot \, |_2$ be two norms on $V$.
Show that the norms are equivalent, meaning that there exist constants $c$ and
$C$ such that
\[
c |x|_1 \leq |x|_2 \leq C|x|_1
\]
for all $x \not= 0$.
\end{prob}

\begin{proof}[Solution]
Let's show that the map $x \mapsto |x|_1$ is continuous with respect to
the metric defined by $|\cdot|_2$.
Let $\varepsilon > 0$.
We want to show that there is a $\delta > 0$ such that $|x|_1 < \varepsilon$
if $|x|_2 < \delta$.

Let's do this by constructing an increasing chain of subspaces of $V$ on
which this is true.
First let $v \not= 0$ and let $L = \kk R v$ be a line.
If $x, x_0 \in L$ then $x = \lambda v$ and $x_0 = \lambda_0 v$ for some
$\lambda, \lambda_0 \in \kk R$, so
\[
|x - x_0|_1
= |\lambda - \lambda_0| |v|_1
= |\lambda - \lambda_0| \frac{|v|_1}{|v|_2} |v|_2
= \frac{|v|_1}{|v|_2} |x - x_0|_2.
\]
If $\varepsilon > 0$ is given we pick $\delta > 0$ such that
$|x - x_0|_2 < \varepsilon |v|_2 / |v|_1$, and get that $|x - x_0|_1 <
\varepsilon$.

Suppose then that $S \subset V$ is a proper subspace on which $|\cdot|_1$ is
continuous and let $v \in V \setminus S$.
We want to show $|\cdot|_1$ is continuous on $S + \kk R v$.
Any vector therein can be written as $x + \lambda v$ with $x \in S$ and
$\lambda \in \kk R$.
Let $v_0 = x_0 + \lambda_0 v$ and $v = x + \lambda v$.
Then
\begin{align*}
|v - v_0|_1
= |(x - x_0) + (\lambda - \lambda_0) v|_1
&\leq |x - x_0|_1 + |\lambda - \lambda_0| |v|_1
\\
&= |x - x_0|_1 + |\lambda - \lambda_0| \frac{|v|_1}{|v|_2} |v|_2
\\
&= |x - x_0|_1 + \frac{|v|_1}{|v|_2} |\lambda v - \lambda_0 v|_2.
\end{align*}
Let $\varepsilon > 0$ be given and let $\delta > 0$ be such that
$|x - x_0|_1 < \varepsilon / 2$
and $|\lambda v - \lambda_0 v|_2 < \varepsilon |v|_2 / (2 |v|_1)$
if $|v - v_0|_2 < \delta$.
Then $|v - v_0|_1 < \varepsilon$, so $|\cdot|_1$ is continuous at $v_0$.

We can thus construct an increasing flag $0 \subset L_1 \subset L_2 \subset
\cdots$ of subspaces of $V$ on which $|\cdot|_1$ is continuous.
As $V$ is finite dimensional, we eventually have $L_n = V$, so $|\cdot|_1$ is
continuous.

Having proven this, we note that the unit sphere is compact since $V$ is
finite dimensional.
As $x \mapsto |x|_1$ is continuous it attains a minimum $c$ and maximum $C$ on
the unit sphere.
If $x \in V$ there exists $u$ in the unit sphere such that $x = |x|_2 u$.
Then
\[
|x|_1 = |x|_2 |u|_1 \leq C |x|_2
\]
for all $x$, and similarly $|x|_1 \geq c |x|_2$.
\end{proof}

\begin{prob}
\begin{enumerate}
\item
Show that the unit ball in a normed vector space is convex.

\item
Show that a linear subspace $E \subset V$ is convex.

\item
If $f : V \to W$ is linear and $E \subset V$ is convex, show that $f(E)$ is convex.
\end{enumerate}
\end{prob}

\begin{proof}[Solution]
All obvious.
\end{proof}

\begin{prob}
Let $V$ and $W$ be normed vector spaces and let $f : V \to W$ be linear.
Show that $f$ is continuous if and only if it is bounded, in the sense
that there exists a constant $C > 0$ such that $|f(x)| \leq C |x|$ for all $x$.
\end{prob}

\begin{proof}[Solution]
Suppose $f$ is continuous.
Then there is a $\delta > 0$ such that $f(B(\delta)) \subset B(1)$, so
$f(B(1)) \subset B(1/\delta)$ by linearity.
If $x \not= 0$ then $(1 / |x| + \varepsilon) x \in B(1)$ for any $\varepsilon > 0$,
so
\[
\frac{1}{|x| + \varepsilon} | f( x ) |
= \biggl| f\biggl( \frac{1}{|x| + \varepsilon} x \biggr) \biggr|
\leq \frac{1}{\delta}
\]
and taking limits we get $|f(x)| \leq |x| / \delta$.

Suppose that $f$ is bounded.
We know that $f$ is continuous if and only if it is continuous at $0$.
Let $\varepsilon > 0$ and let $\delta < \varepsilon/C$; then
\[
|f(x)| \leq C |x| < \varepsilon
\]
for any $x \in B(\delta)$, so $f$ is continuous at $0$.
\end{proof}

\begin{prob}
Let $V$ be a normed vector space.
We are going to prove that $V$ can be embedded in a Banach space.
\begin{enumerate}
\item
Define $X$ to be the set of Cauchy sequences in $V$.
Show that $X$ may be given the structure of a vector space.

\item
Show that $(x_n) \mapsto \lim_{n \to \infty} |x_n|$ defines a seminorm on $X$;
that is a function that satisfies the conditions to be norm except $|x| = 0$
does not imply $x = 0$.

\item
Show that $N := \{ x \in X \mid |x| = 0 \}$ is a subspace of $X$, and that
the seminorm on $X$ induces a norm on the quotient space $X / N$.

\item
Show that $X / N$ is complete, and thus a Banach space.

\item
Show that the map $f : V \to X / N$ that sends $x$ to the image of the sequence
$(x,x,\ldots)$ under the quotient map is linear, injective, and continuous.

\item
Show that $X / N$ satisfies the following universal property:
If $Y$ is a Banach space and $f : V \to Y$ is a continuous linear map, then
there is a unique continuous linear map $\hat f : X/N \to Y$ such that the
following diagram commutes:
\[
\begin{tikzcd}
V \ar[dr,"f"] \ar[d] &
\\
X/N \ar[r,"\hat{f}"] & Y
\end{tikzcd}
\]
\end{enumerate}
\end{prob}


\begin{proof}[Solution]
1. We have $X = \{ (x_n) \mid x_n \in V \text{ is Cauchy} \}$.
Define $\lambda x = (\lambda x_n)$ and $x + y = (x_n + y_n)$.
The first is clearly well defined and the second is also because
\[
| x_n + y_n - (x_m + y_m) |
= | (x_n - x_m) + (y_n - y_m) |
\leq |x_n - x_m| + |y_n - y_m|
\]
so $(x_n + y_m)$ is Cauchy if $(x_n)$ and $(y_n)$ are.
Then $X$ is a vector space as it is a subspace of the space of all sequences in
$V$, which is just $\prod_{\kk N} V$.

2. First off this is well-defined because if $(x_n)$ is Cauchy then so is
$(|x_n|)$ and $\kk R$ is complete.
This behaves correctly with respect to scaling and satisfies the triangle
inequality.
However $|x| = 0$ only implies that $(x_n)$ converges to $0$, not that it is
zero, so this is a seminorm.

3. Since $|\cdot|$ is a seminorm we see that $N = \{ x \in X \mid |x| = 0\}$
is a linear subspace.
We attempt to define a norm on $X / N$ by $|[x]| = |x|$.
Let $[y] = 0$. Then
\[
|[x + y]|
= |x + y|
\leq |x| + |y|
= |x|
= |[x]|
\]
and
\[
|[x]|
= |x|
= |x + y - y|
\leq |x+y| + |y|
= |x + y|
= |[x + y]|
\]
so $|\cdot|$ is well-defined on $X / N$.

This satisfies $|\lambda [x]| = |\lambda| |[x]|$
and
\[
|[x] + [y]|
= |[x + y]|
= |x + y|
\leq |x| + |y|
= |[x]| + |[y]|.
\]
If $|[x]| = 0$ then $|x| = 0$ so $x \in N$ and $[x] = 0$.
We thus have a norm.

4. Let $(x_n)$ be a Cauchy sequence in $X/N$. Then there are $x_{nm}$ in $V$ such
that $x_n = x_{mn}$, and each $(x_{mn})_m$ is a Cauchy sequence.
Let $y = (x_{nn})$.
Then
\[
x_{nn} - x_{mm}
= (x_{nn} - x_{nl})
+ (x_{nl} - x_{ml})
+ (x_{ml} - x_{mm})
\]
each of which we can make $< \varepsilon/3$, so $(x_{nn})$ is Cauchy.
Similarly we see that $(x_n) \to y$, so $X$ is complete.

5. The map is clearly linear and injective.
We have $|(x,x,\ldots)| = |x|$, so it is bounded and thus continuous.

6. Let $Y$ be a Banach space and let $f : V \to Y$ be continuous.
We define $\hat f : X/N \to Y$ by
\[
\hat f([x]) = \lim_{n\to \infty} f(x_n),
\]
where $[x] = (x_n)$ is the image of a Cauchy sequence in $V$.
This is well-defined because $(x_n)$ is a Cauchy sequence in $V$ and $f$ is
continuous, so $(f(x_n))$ is a Cauchy sequence in $Y$ and thus has a limit
as $Y$ is Banach.
Further, if $x_n \to 0$ then $f(x_n) \to 0$, so this maps all of $N$ to $0$.
We see that $\hat f$ is a linear map.

As $f : V \to Y$ is continuous there is a $C > 0$ such that $|f(x)| \leq C |x|$
for all $x \in V$.
Let $[x] \in X / N$ and let $(x_n)$ be a Cauchy sequence that represents $[x]$.
Then $|f(x_n)| \leq C |x_n|$ for all $n$, so
\[
|\hat f([x])|
= \biggl| \lim_{n \to \infty} f(x_n) \biggr|
= \lim_{n \to \infty} | f(x_n) |
\leq C \lim_{n \to \infty} |x_n|
= C |[x]|
\]
where we can pull the limit out because both limits exist.
Then $\hat f$ is bounded and thus continuous.
\end{proof}

\end{document}
