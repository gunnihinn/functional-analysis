\documentclass[11pt]{article}

\usepackage{tgpagella}
\linespread{1.0}
\usepackage[utf8]{inputenc}
\usepackage[T1]{fontenc}

\usepackage[normalem]{ulem}
\usepackage{textcomp}
\usepackage{hyperref}

\usepackage{amsmath}
\usepackage{amssymb}
\usepackage{amsthm}

\usepackage{tikz-cd}
\usepackage{color}

\newtheorem*{theo}{Theorem}
\theoremstyle{definition}
\newtheorem{prob}{Problem}
\newtheorem*{defi}{Definition}

\newcommand{\kk}[1]{\mathbf{#1}}
\newcommand{\cc}[1]{\mathcal{#1}}

\def\<{\langle}
\def\>{\rangle}
\def\norm#1{\| #1 \|}

\DeclareMathOperator{\Hom}{Hom}
\DeclareMathOperator{\supp}{supp}
\DeclareMathOperator{\im}{im}
\DeclareMathOperator{\id}{id}
\DeclareMathOperator{\codim}{codim}
\DeclareMathOperator{\ann}{ann}

\def\head{
\begin{center}
\textbf{\LARGE Homework set 6}
\end{center}
\medskip
}

\begin{document}

\head

A \emph{hyperplane} in a normed space $V$ is a set of the form $H = x_0 + H_0$,
where $H_0 \subset V$ is a subspace such that $\codim H_0 := \dim V / H_0 = 1$.

\begin{prob}
Let $V$ be a normed space.
Show that $H$ is a closed hyperplane if and only if there exists a bounded
linear functional $f \in V^\vee$ such that $H = f^{-1}(c)$ for some $c \in \kk R$.
Then show that $V \setminus H$ consists of two disjoint connected components.
\end{prob}

\begin{proof}[Solution]
Let $H = x_0 + H_0$ be a closed hyperplane.
Then $H_0 \subset V$ is a closed subspace, so there exists a bounded linear
functional $f \in V^\vee$ such that $H_0 \subset \ker f$.
We then have the exact sequences
\[
\begin{tikzcd}
0 \ar[r] & H_0 \ar[r] \ar[d,"j"] & V \ar[r,"q"] \ar[d,"\id"] & V / H_0 \ar[r] & 0
\\
0 \ar[r] & \ker f \ar[r] & V \ar[r,"f"] & \kk R \ar[r] & 0
\end{tikzcd}
\]
from which we define a linear map $p : V/H_0 \to \kk R$ by $[x] \mapsto f(x)$,
where $x \in [x] \in V / H_0$. This map is linear and nonzero, and thus an
isomorphism.
It follows that $j$ is also an isomorphism.
Therefore $H_0 = \ker f$, so $H = f^{-1}(f(x_0))$.

Let now $f \in V^\vee$. If $x$ is such that $f(x) = c$ then $f^{-1}(c) = x +
\ker f$, so it is a hyperplane, and closed because $\ker f = f^{-1}(0)$ is closed.

The complement is
\[
V \setminus H
= f^{-1}((-\infty, c) \cup (c, \infty))
= f^{-1}((-\infty, c)) \cup f^{-1}((c, \infty))
=: V_{-} \cup V_{+}
\]
which are two open disjoint sets.
If $x, y \in V_{+}$ then
\[
f((1-t) x + t y)
= (1-t) f(x) + t f(y)
> (1-t) c + t c = c
\]
so the segment $\{(1-t) x + ty \mid 0 \leq t \leq 1 \} \subset V_+$. The
set is thus path connected.
\end{proof}

\begin{prob}
Let $V$ be a normed space and let
\begin{align*}
S(r) = \{ x \in V \mid \norm{x} = r \},
\quad
B(r) = \{ x \in V \mid \norm{x} < r \},
\end{align*}
be the sphere and ball of radius $r$.
Show that for any $x_0 \in S(r)$ there exists a hyperplane $H \subset V$ that
contains $x_0$ such that the open ball $B(r)$ is
entirely contained in one component of $V \setminus H$.
\end{prob}

\begin{proof}[Solution]
Let $S = \kk R x_0$ and define $f : S \to \kk R$ by $f(x_0) = r$,
where $r = |x_0|$, and extend by linearity.
Then
\[
|f|
= \sup_{|\lambda x_0| = 1} |f(\lambda x_0)|
= \sup_{|\lambda| r = 1} |\lambda| r
= 1.
\]
By Hahn--Banach we can extend this to $f \in V^\vee$ such that $|f| = 1$.
For any $x \in B(r)$ we then have
\[
|f(x)| \leq |f| |x| < r,
\]
so $B(r) \subset \{ x \in V \mid f(x) < r \}$ while
$x_0 \in \{ x \in V \mid f(x) = r \}$.
\end{proof}

The \emph{annihilator} of a set $M$ in a normed space $V$ is the set
\[
M^{\perp} = \{ f \in V^\vee \mid \text{$f(x) = 0$ for all $x \in M$} \}.
\]

\begin{prob}
Let $V$ be a normed space and $S \subset V$ a subspace.
Show that $S^\vee$ is isometric to $V^\vee / S^\perp$.
\end{prob}

\begin{proof}[Solution]
We have short exact sequences
\[
\begin{tikzcd}
0 \ar[r] & \ker \rho \ar[r] & V^\vee \ar[d] \ar[r,"\rho"] & S^\vee \ar[r] & 0
\\
0 \ar[r] & S^\perp \ar[r] & V^\vee \ar[r] & V^\vee / S^\perp \ar[r] & 0,
\end{tikzcd}
\]
where the upper row is exact because of Hahn--Banach.
The map $\rho : V^\vee \to S^\vee$ is just the restriction map $f \mapsto f_{|S}$.
Its kernel is
\[
\ker \rho
= \{ f \in V^\vee \mid f_{|S} = 0 \}
= S^\perp.
\]
We thus get a well-defined linear map $\psi : S^\vee \to V^\vee / S^\perp$
by $\psi(f) = [\hat f]$, where $\hat f$ is an extension of $f$.

This map is injective, because if $\psi(f) = [ \hat f ] = 0$ then $\hat f \in
S^\perp$ and $f = \rho(\hat f) = 0$.
It is also surjective: Let $[g] \in V^\vee / S^\perp$, where $g \in V^\vee$.
Set $f = g_{|S}$. Then $\psi(f) = [g]$.

Note that every $x \in V$ defines a linear functional $e_x \in (V^\vee)^\vee$
by $e_x(f) = f(x)$.
We have $S^\perp = \bigcap_{x \in S} \ker e_x$, so $S^\perp$ is the intersection
of closed sets and thus closed.
Therefore $V^\vee / S^\perp$ is a normed space.
For $f \in S^\vee$ we have
\[
|\psi(f)|
= |[ \hat f ]|
= \inf_{\hat f_{|S} = f} |\hat f|
= |f|,
\]
where the last equality is by Hahn--Banach.
\end{proof}

\begin{prob}
Let $V$ be a normed space and $S \subset V$ a closed subspace.
Show that $(V / S)^\vee$ is isometric to $S^\perp$.
\end{prob}

\begin{proof}[Solution]
Since $S$ is closed the quotient $V/S$ is a normed space.
We define a map $\psi : S^\perp \to (V/S)^\vee$ by
\[
\psi(f)([x]) = f(x).
\]
We have to show this is a well-defined linear map and that $\psi(f)$ is
bounded. By definition we have $f_{|S} = 0$, so the map is well-defined and
linear.

This map is injective: If $\psi(f)([x]) = 0$ for all $[x]$ then $f(x) = 0$ for
all $x$, so $f = 0$.
The map is also surjective: Let $g \in (V/S)^\vee$.
Define a map $f \in V^\vee$ by $f(x) = g([x])$.
Then $f_{|S} = 0$ so $f \in S^\perp$ and $\psi(f) = g$.

Now
\[
|\psi(f)|
= \sup_{[x] = 1} |\psi(f)([x])|.
\]
We have $|[x]| = \inf_{y \in S}|x - y|$, so if $|[x]| = 1$ there exists
$x \in V$ and $(y_n)$ in $S$ such that $|x - y_n| \to 1$ as $n \to \infty$ (and
$|x - y| \geq 1$ for all $y \in S$).
For such an $x$ we have
\[
|\psi(f)([x])|
= |f(x)|
= \frac{|f(x)|}{|x|} |x|
\leq |f| \, |x|
\leq |f|(|y| + |x - y|)
\]
for $y \in S$.
Taking $\inf$ over $y \in S$ we get $|\psi(f)([x])| \leq |f|$ when $|[x]| = 1$,
so $|\psi(f)| \leq |f|$.

For any $x$ we have $|[x]| \leq |x|$, so
\[
|\psi(f)|
= \sup_{|[x]|\not= 0} \frac{|\psi(f)([x])|}{|[x]|}
\geq \sup_{x\not\in S} \frac{|f(x)|}{|x|}
= \sup_{x \not= 0} \frac{|f(x)|}{|x|}
= |f|
\]
because $f_{|S} = 0$, so $|\psi(f)| = |f|$.
\end{proof}

\end{document}
