\documentclass[11pt]{article}

\usepackage{tgpagella}
\linespread{1.0}
\usepackage[utf8]{inputenc}
\usepackage[T1]{fontenc}

\usepackage[normalem]{ulem}
\usepackage{textcomp}
\usepackage{hyperref}

\usepackage{amsmath}
\usepackage{amssymb}
\usepackage{amsthm}

\usepackage{tikz-cd}
\usepackage{color}

\newtheorem*{theo}{Theorem}
\theoremstyle{definition}
\newtheorem{prob}{Problem}
\newtheorem*{defi}{Definition}

\newcommand{\kk}[1]{\mathbf{#1}}
\newcommand{\cc}[1]{\mathcal{#1}}

\def\<{\langle}
\def\>{\rangle}
\def\norm#1{\| #1 \|}

\DeclareMathOperator{\Hom}{Hom}
\DeclareMathOperator{\supp}{supp}
\DeclareMathOperator{\im}{im}
\DeclareMathOperator{\codim}{codim}
\DeclareMathOperator{\ann}{ann}

\def\head{
	\begin{center}
		\textbf{\LARGE Homework set 9}
		\\[3pt]
		Due by 15:00 on Monday, October 23, 2023.
	\end{center}
	\medskip
}

\begin{document}

\head

Please select \textbf{three} problems to solve and hand in written solutions
either in person or to \verb+gunnar@magnusson.io+.

\begin{prob}
	Let $f : V \to V$ be a self-adjoint bounded operator on a Hilbert space.
	\begin{enumerate}
		\item
			Suppose that $\lambda$ is an eigenvalue of $f$. Show that $\lambda$ is real.

		\item
			Show that the eigenvectors corresponding to two different eigenvalues are orthogonal.
	\end{enumerate}
\end{prob}

\begin{prob}
	Let $V = L^2([0,1])$ and let $T : V \to V$ be the operator $T(f)(x) = x f(x)$.
	Show that $T$ is self-adjoint, but that $T$ has no eigenvalues.
\end{prob}

\begin{prob}
	Let $V$ be a Hilbert space and $A : V \to V$ a bounded operator.
	Let $f(z) = \sum_{n \geq 0} a_n z^n$ be a power series with radius of convergence $R > 0$.
	If $|A| < R$, show that there exists a bounded operator $T$ on $V$ such that for all $x,y \in V$ we have
	\[
		\< T(x), \bar y \> = \sum_{n \geq 0} a_n \< A^n x, \bar y \>.
	\]
	(The operator $T$ is usually denoted $f(A)$.)
\end{prob}

\begin{prob}
	Calculate the adjoints of the left- and right-shift operators
	\begin{align*}
		L(x_1, x_2, \ldots) & = (x_2, x_3, \ldots),
		\\
		R(x_1, x_2, \ldots) & = (0, x_1, x_2, \ldots),
	\end{align*}
	on $\ell^2$.
\end{prob}

\begin{prob}
	Calculate $\exp R$ of the right-shift operator $R$ on $\ell^2$.
\end{prob}


\end{document}



























