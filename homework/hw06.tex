\documentclass[11pt]{article}

\usepackage{tgpagella}
\linespread{1.0}
\usepackage[utf8]{inputenc}
\usepackage[T1]{fontenc}

\usepackage[normalem]{ulem}
\usepackage{textcomp}
\usepackage{hyperref}

\usepackage{amsmath}
\usepackage{amssymb}
\usepackage{amsthm}

\usepackage{tikz-cd}
\usepackage{color}

\newtheorem*{theo}{Theorem}
\theoremstyle{definition}
\newtheorem{prob}{Problem}
\newtheorem*{defi}{Definition}

\newcommand{\kk}[1]{\mathbf{#1}}
\newcommand{\cc}[1]{\mathcal{#1}}

\def\<{\langle}
\def\>{\rangle}
\def\norm#1{\| #1 \|}

\DeclareMathOperator{\Hom}{Hom}
\DeclareMathOperator{\supp}{supp}
\DeclareMathOperator{\im}{im}
\DeclareMathOperator{\codim}{codim}
\DeclareMathOperator{\ann}{ann}

\def\head{
\begin{center}
\textbf{\LARGE Homework set 6}
\\[3pt]
Due by 15:00 on Monday, October 2, 2023.
\end{center}
\medskip
}

\begin{document}

\head

Please select \textbf{three} problems to solve and hand in written solutions
either in person or to \verb+gunnar@magnusson.io+. You may quote problems from
older homework sets and results we've read in the textbook if you feel like
they help.

A \emph{hyperplane} in a normed space $V$ is a set of the form $H = x_0 + H_0$,
where $H_0 \subset V$ is a subspace such that $\codim H_0 := \dim V / H_0 = 1$.

\begin{prob}
Let $V$ be a normed space.
Show that $H$ is a closed hyperplane if and only if there exists a bounded
linear functional $f \in V^\vee$ such that $H = f^{-1}(c)$ for some $c \in \kk R$.
Then show that $V \setminus H$ consists of two disjoint connected components.
\end{prob}

\begin{prob}
Let $V$ be a normed space and let
\begin{align*}
S(r) = \{ x \in V \mid \norm{x} = r \},
\quad
B(r) = \{ x \in V \mid \norm{x} < r \},
\end{align*}
be the sphere and ball of radius $r$.
Show that for any $x_0 \in S(r)$ there exists a hyperplane $H \subset V$ that
contains $x_0$ such that the open ball $B(r)$ is
entirely contained in one component of $V \setminus H$.
\end{prob}

The \emph{annihilator} of a set $M$ in a normed space $V$ is the set
\[
M^{\perp} = \{ f \in V^\vee \mid \text{$f(x) = 0$ for all $x \in M$} \}.
\]
We say that a normed space $V$ is \emph{isometric} to a normed space $W$ if
there exists a linear isomorphism $f : V \to W$ that preserves their norms,
that is, such that $\norm{f(x)} = \norm{x}$ for all $x \in V$.

Recall that if
\[
0 \longrightarrow S \longrightarrow V \longrightarrow V/S \longrightarrow 0
\]
is a short exact sequence of vector spaces, then
\[
0 \longrightarrow (V/S)^* \longrightarrow V^* \longrightarrow S^* \longrightarrow 0
\]
is also exact.
The next two problems establish a similar duality between sub- and quotient
spaces for normed spaces and bounded linear functionals.

\begin{prob}
Let $V$ be a normed space and $S \subset V$ a subspace.
Show that $S^\vee$ is isometric to $V^\vee / S^\perp$.\footnote{Why does the quotient space have a norm and not just a seminorm?}
\end{prob}

\begin{prob}
Let $V$ be a normed space and $S \subset V$ a closed subspace.
Show that $(V / S)^\vee$ is isometric to $S^\perp$.
\end{prob}

\end{document}
